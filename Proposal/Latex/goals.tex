As can be seen in the previous section, the problem with most current solutions is feasibility. Even though they guarantee security against different kinds of adversaries, they are not suited to be used in real world applications because of the operational time the different measures incur.\\
To provide a solution which is more applicable in reality without sacrificing security, Doerner et al. [TODO Zitat] introduce a new oblivious data strucure called "Oblivious Linked Multi-List". The data structure uses data arrays combined with a method by Zahur et al. [TODO Zitat] to obliviously permutate them. This provides the means to obliviousy accessing data, thus hiding access patterns and the derivable information. Even though the authors claim that usage of the Oblivious Linked Multi-List ensures security in the honest-but-curious attacker model, it has not been formally proven yet. The goal of this thesis is to formalize the rough intuition given by Doerner et al. [TODO Zitat] and provide a formal proof based on the Simulation Technique introduced by Goldreich [TODO richtig aufschreiben/Zitat]. It was later explained by Lindell in an extensive Tutorial [TODO Zitat] which will also be used as a basis for this thesis.\\
To do so the whole approach will first be sectioned into it's different components such as the Oblivious Linked Multi-List and the Gale Shapley algorithm using it. After proving security for both parts seperately we will achieve security for the overall construction.
\ \\With regard to optional topics I see (there are) two possible scenarios. 
The first one is that in the course of working with the presented protocol and it's proof of security, it becomes clear that the assumption about the protocols security in context of the presented security model doesn't hold. In this case the first step would be to examine and explain the problems that arise. The insights gained could be used to identify a different, weaker security model that solves the problems the initial protocol implied. With such an altered security model a proof can then again be attempted.\\
The second one is that the proof turns out to be more straight forward than expected. A reasonable course of action could be to go a step further and investigate how the proven security transfers to other security models, such as an active attacker scenario.