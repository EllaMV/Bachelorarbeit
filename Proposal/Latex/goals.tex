As I already mentioned, the solution proposed by Doerner et al. is claimed to be secure but it's security has not been formally proven yet. The goal of this thesis is to formalize the rough intuition given by Doerner et al. [TODO Zitat] and provide a formal proof based on the Simulation Technique introduced by Goldreich [TODO richtig aufschreiben/Zitat]. It was later explained by Lindell in an extensive Tutorial [TODO Zitat] which will also be used as a basis for this thesis.\\
First I will take a closer look at the oblivious linked multi-list. According to the simulation technique I will establish an ideal model of it's functionality. The model includes a formal description of what the functionality takes as input and what the expected output is. Then I will examine how Doerner et al. realize this functionality in detail. The questions answered will include: How do they transform the input? Which operations are performed on the data? I will then proceed to formally proof security including correctness and privacy. This will be done by showing that it works correctly, meaning that their output equals the ideal model's output and showing that nothing can be learned about the input, respectively.\\
After proving security for the oblivious linked multi-list I will use it as a blackbox when showing the whole protocol's security. To prove the protocol's security I will carry out the same steps as I did with the oblivious linked multi-list, i.e. establishing an ideal model, examining Doerner et al.'s construction and then formally proving security. By showing that the protocol is secure when using the oblivious linked multi-list (which was previously proven to be secure) I can conclude that the combination of the algorithm and oblivious linked multi-lists is also secure.\\
Another advantage that this modular approach has, becomes relevant when the protocol turns out not to be secure in the specified setting. Splitting the protocol into two primitives allows me to point out where things go wrong more specifically and adjust the attacker model to potentially show security in a weaker security model.\\
It is also possible that the proof turns out to be more straight forward than expected. A reasonable course of action could be to go a step further and investigate how the proven security transfers to other security models, such as an active attacker scenario. Here again I would be making use of the modular approach, benefiting from it's advantages.\\