As already mentioned, SSM is claimed to be secure but its security has not been formally proven yet. The goal of this thesis is to formalize the rough intuition given by \citeauthor{smas} and provide a formal proof based on the real/ideal world paradigm using the simulation technique introduced by \citet{foc}. The Simulation Technique was later described by \citet{htsi} in an extensive Tutorial which will also be used as a basis for this thesis. The simulation paradigm states the following: We have a functionality we want to compute and a protocol realizing the functionality. In addition to this we model the functionality as an ideal functionality, i.e., a formal model of the functionality's expected behaviour in an ideal world with a trusted third party. If we can simulate what an (honest-but-curious) attacker would see during the execution of the protocol in the real world with real participants, indistinguishably from what they see in the ideal world, the protocol is as secure as computing the functionality with the help of an ideal trusted party.\\
First, we will consider the OLML. According to the simulation technique we will model the notion of an OLML as an ideal functionality. Then we will examine how \citeauthor{smas} realize this functionality in detail, by answering how they transform the input and which operations are performed on the data. We will then proceed to formally prove security including correctness and privacy. This will be done by showing that it works correctly, meaning that their output equals the ideal model's output and showing that nothing (beyond the function's output) can be learned about the input, respectively. The proof for privacy is based on the simulation paradigm as described above.\\
After proving security of their proposed OLML implementation we will use it as a blackbox when showing the whole protocol's security. To prove the protocol's security, we will carry out the same steps as we did with the OLML, i.e. establishing an ideal model of a stable matching functionality, examining \citeauthor{smas}'s construction and then formally proving security. By showing that the protocol is secure when using the OLML (which was previously proven to be secure) as an ideal blackbox we can conclude that the combination of the algorithm and OLML is also secure, using the well-known composition theorem by \citet{foc}.\\
Another advantage this modular approach has becomes relevant when the protocol turns out not to be secure in the specified setting. Splitting the protocol into two parts allows us to point out where things go wrong more specifically and adjust the attacker model to potentially show security in a weaker security model or adjust the protocol itself.\\
It is also possible that the proof turns out to be more straight forward than expected. A reasonable course of action could be to go a step further and investigate how the proven security transfers to other security models, such as an active attacker scenario. Here again, we would be making use of the modular approach, benefiting from its advantages.\\