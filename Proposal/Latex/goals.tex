As already mentioned, the solution proposed by Doerner et al. is claimed to be secure but it's security has not been formally proven yet. The goal of this thesis is to formalize the rough intuition given by Doerner et al. \todo{Zitat, Doerner} and provide a formal proof based on the Simulation Technique introduced by Goldreich \todo{Zitat, vermutl. in Tutorial}. The Simulation Technique was later explained by Lindell in an extensive Tutorial \todo{Zitat, How to Simulate} which will also be used as a basis for this thesis.\\
First we will consider the oblivious linked multi-list. According to the simulation technique we will model the notion of an oblivious linked multi-list as an ideal functionality, i.e. a formal model of the oblivious linked multi-list's expected behaviour. Then we will examine how Doerner et al. realize this functionality in detail. The questions answered will include: How do they transform the input? Which operations are performed on the data? We will then proceed to formally prove security including correctness and privacy. This will be done by showing that it works correctly, meaning that their output equals the ideal model's output and showing that nothing (beyond the funcion's output) can be learned about the input, respectively. The proof for privacy is based on the simulation paradigm\todo{Zitat, vermutl. Tutorial}, which states the following: Assume it was proven that nothing can be learned about the private inputs in the ideal model. If indistinguishability between the real and ideal world's execution can then be shown, it can be concluded that nothing can be learned about the private inputs in the real world either.\\
After proving security of their proposed oblivious linked multi-list implementation we will use it as a blackbox when showing the whole protocol's security. To prove the protocol's security we will carry out the same steps as we did with the oblivious linked multi-list, i.e. establishing an ideal model of a stable matching functionality, examining Doerner et al.'s construction and then formally proving security. By showing that the protocol is secure when using the oblivious linked multi-list (which was previously proven to be secure) as an ideal blackbox we can conclude that the combination of the algorithm and oblivious linked multi-lists is also secure, using the well known composition theorem by Goldreich.\todo{Zitat, vermutl. Tutorial}\\
Another advantage that this modular approach has, becomes relevant when the protocol turns out not to be secure in the specified setting. Splitting the protocol into two parts allows us to point out where things go wrong more specifically and adjust the attacker model to potentially show security in a weaker security model.\\
It is also possible that the proof turns out to be more straight forward than expected. A reasonable course of action could be to go a step further and investigate how the proven security transfers to other security models, such as an active attacker scenario. Here again we would be making use of the modular approach, benefiting from it's advantages.\\