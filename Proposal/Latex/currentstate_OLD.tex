As mentioned above there are general purpose frameworks which are applicable to arbitrary functions but are not efficient enough to be used in practice. Over the years technical advances allowed for more feasible implementations leading to the development of various products, including tools like EMP-toolkit, Obliv-C or ObliVM. Even though there were huge improvements there are still limitations. In addition to efficiency problems Hastings et al. \todo{Zitat} for example point out lack of documentation and correctness erorrs \todo{Zitat}. \\
Because of this, general purpose frameworks are not a desirable solution, as well as outsourcing computation due to reasons explained above.\\
To avoid this, even prior to Doerner et al. \todo{Zitat} specific-purose protocols were developed, which compute a stable matching and are carried out between the involved parties only. As mentioned above as well, in many cases outsourcing the computation to a commonly trusted party is not a good option either. Stable matching in the context of multiparty-computation comprises multiple different aspects, such as data access strategies or the underlying algorithm making it an interesting problem both in theory and practice. Both of these are research topics of their own with several possible approaches and solutions. Naturally this yields various possibilities of combining them to solve the problem of securely computing a stable matching. As opposed to the discussed approach by Doerner et al. \todo{Zitat} which uses the Gale-Shapley algorithm, Blanton et al. \todo{Zitat} for example use Breadth First Search. \\
Some more examples of work prior to Doerner et al. \todo{Zitat} are the protocols of Golle \todo{Zitat} with roughly O$(n^5)$ public-key operations \todo{formatieren} or the previously best implementation of the Gale-Shapley algorithm with a runtime of over 33 hours, working on sets of only 512 members each \todo{Zitat}.