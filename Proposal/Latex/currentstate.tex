Stable matching in the context of multiparty-computation comprises multiple different aspects, such as data access strategies, the realized algorithm, or the attacker model, making it an interesting problem both in theory and practice. All of these are research topics of their own with several possible approaches and solutions. Naturally this yields various possibilities of combining them to solve the problem of securely computing a stable matching. As opposed to the discussed approach by Doerner et al. \citep{smas} which uses the Gale-Shapley algorithm, Blanton et al. \citep{dogafscao} for example use a form of Breadth First Search.\\
\ \\As for the Gale-Shapley algorithm there are several existing approaches. 
\ \\The usability problem of generic solutions for stable matchings, in particular the Gale-Shapley algorithm, was also mentioned by Golle \citep{apsma}. To hide the index of the accessed array element, thus making the access appear oblivious, generic solutions produce effort linear in the number of elements. Applying this to array access intense algorithms such as stable matching algorithms results in a lot of overhead. Due to this, he proposed a \glqq practical\grqq{} solution with complexity of supposedly $\mathcal{O}(n^3)$ asymmetric cryptographic operations. It is based on matching authorities, and ensured to fulfil privacy and correctness requirements when semi-honesty can be guaranteed for a majority of performing matching authorities. \\
Problems with Golle's solution, especially a complexity of actually $\mathcal{O}(n^5)$, were later found by Franklin et al. \citep{iefpsm}, resulting in two new solutions. One is based on Golle's approach and the other one is based on garbled circuits and a protocol by Naor Nissim for secure function evaluation, both with $\mathcal{O}(n^4)$ public key operations and computation complexity respectively.\\
Previously to Doerner et al. the best approach was proposed by Zahur et al. \citep{rsqo}. With a runtime of roughly 33 hours on input of 512 x 512 participants, it is over 40 times slower than Doerner et al.'s protocol. 