Secure multiparty-computation (MPC) is a concept for securely computing (potentially probabilistic) functions with multiple parties involved. Each party has private input which must not be disclosed to the other parties. Based on these, the different parties jointly execute a protocol to obtain the function's result. In this context security refers to the privacy of parties' inputs and correctness. To achieve privacy the protocol must ensure that no information about one party's input is leaked to any other parties. Correctness is attained when the computed result equals the functions output based on the given input.\\
\ \\Since there are many areas of application where MPC would be useful, general purpose frameworks were developed early on. With these, arbitrary functions can be converted into a format which is applicable to MPC protocols. These are then executed on for example boolean or arithmetic curcuits \todo{Zitat}.\\ 
Even though the idea of these has been around for many decades already and it has been proven that they fulfill security requirements they are not used as much as one would expect. \todo{Zitat} The main reason for this is that for many interesting problems they are not efficient enough to be used in practice when solved with general purpose frameworks. \todo{Zitat} For these problems special-purpose algorithms with faster execution time can be constructed. This allows them to be used in practice but involves much more work because there is no single solution that fits all as it is the case with general purpose frameworks. \\
\ \\One such problem is finding a stable matchhing. Matching algorithms are used to match members of two different sets to each other with regards to their respective preferences. A stable matching is achieved when there are no two members (one from each set) that aren't matched to each other, such that they would rather be matched to each other than to their assigned matches.\\ 
Such algorithms are used to for example match residents to residency programs or students to public schools \todo{Zitat}. Obviously the best solution would be that every member is assigned to their first preference, but in reality this is rarely possible. Stable matchings are a desirable solution in such cases to still achive a high level of member satisfaction and in some sense fairness.\\
One solution for computing a stable matching is the Gale-Shapley algorithm, developed by David Gale and Lloyd Shapley \todo{Zitat}, which will also be the one we are interested in. To provide a solution which is more applicable in reality without sacrificing security, Doerner et al. \todo{Zitat} introduce a new oblivious data strucure called "Oblivious Linked Multi-List" and make use of it during the execution of the Gale-Shapley algorithm. The data structure uses data arrays combined with a method by Zahur et al. \todo{Zitat} to obliviously permutate them. This provides the means to obliviously access data, thus hiding access patterns and the derivable information. The authors claim that using the Gale-Shapley algorithm combined with the Oblivious Linked Multi-List ensures security in the honest-but-curious attacker model. Honest-but-curious attackers don't acitvely intefere with the protocol. They follow instructions correctly and act as expected but try to learn more from the execution than they are supposed to. This could for example be learning about one party's preferences based on the order they are accessed in. Even though the authors give a rough intuition provided in the form of explaining arguments as to why security holds in this setting, it has not been formally proven yet. \citeauthor{soktest}\\
\cite{sok}