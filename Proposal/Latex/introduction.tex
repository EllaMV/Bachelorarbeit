This template contains some hints of what to write in each section.
\emph{This is just an example.} There are good reasons to deviate from it. These are just some ideas to get started. 

For the introduction:
\begin{itemize}
  \item What general topic are we interested in, why does it matter (e.g., where is it used, what problem does it solve)?
  \item Explain some more background for the topic, help the reader (who may have never heard of the topic) understand what you are talking about.
\end{itemize}
(Tip: make such a bullet point list for yourself (with more concrete talking points) for all sections, \emph{then} start writing text)

Secure multiparty-computation (MPC) is a concept for securely computing functions with multiple parties involved. In this context security refers to the different parties' private inputs. The goal is to collectively execute a protocol and as a result obtain the wanted output without the necessity to reveal private data. \\
Secure multiparty-computation yields many practical areas of application, one being stable matching algorithms. Matching algorithms are used to match members of two different sets to each other with regards to their respective preferences. Such algorithms are used to for example match residents to residency programs or students to public schools [TODO Zitat]. A stable matching is achieved when there are no two members (one from each set) that aren't matched to each other, such that they would rather be matched to each other than to their assigned matches. Obviously the best solution would be that every member is assigned to their first preference, but in reality this is rarely possible. Stable matchings are a desirable solution in such cases to still achive a high level of member satisfaction and in some sense fairness.\\
One solution for computing a stable matching is the Gale-Shapley algorithm, developed by David Gale and Lloyd Shapley [TODO Zitat].  